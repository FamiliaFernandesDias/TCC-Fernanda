
% resumo em inglês
\begin{resumo}[Abstract]
	\begin{otherlanguage*}{english}
		% A problem-based active methodology is an educational approach that places students at the center of the learning process, encouraging them to solve real-world problems through the application of knowledge and skills. This methodology is particularly effective as it promotes active participation, critical thinking, creativity, and collaboration. The research seeks to determine whether this methodology constitutes an effective alternative path for constructing spatial geometric concepts. The objectives include developing problem-solving skills, deep understanding of geometric concepts, practical application of geometry, development of transversal competencies, and increased student engagement and motivation. Teachers act as facilitators, guiding students through practical and collaborative activities such as case studies, simulations, group projects, debates, and practical experiences. The methodology includes steps such as problem identification, contextualization, exploration, collaboration, experimentation, reflection, and presentation. The dissertation is organized into chapters that cover the introduction, theoretical framework, methodological aspects, development of didactic activities, and final considerations.

		It is noticed that, in the teaching of Spatial Geometry, many students have difficulties in understanding abstract concepts, such as those related to polyhedrons, and while applying them in practical contexts. The traditional approach, often focused on memorizing formulas and procedures, does not seem to promote a deep and meaningful understanding of the content. Therefore, in this work, we propose to verify whether the use of the Teaching-Learning-Assessment Methodology of Mathematics through Problem Solving \cite{resolucaoDeProblemas2019, polya1978} constitutes an alternative path for the construction of spatial geometric concepts and content by High School students. It is an educational approach that places the student at the center of the learning process, encouraging them to be an active and autonomous learner \cite{polya1978}. In this method, students learn by solving real or simulated problems, instead of simply receiving information passively. This approach was applied to students in the sophomore year of High School in a public school. At the end of this research, we were able to demonstrate that problem solving is one of the main methods for teaching, learning, and assessing Mathematics in the classroom. Based on the evidence collected in this research, we strongly believe that the Teaching-Learning-Assessment Methodology of Mathematics through Problem Solving is an effective alternative that allows students to construct mathematical concepts and content, exploring and taking advantage of their own potential and skills. The students were committed to the work and focused on the proposed activities, with no dispersion. We also achieved a significant evolution in qualitative and quantitative performance. Given the results, it is possible to conclude that the Problem Solving Methodology thus becomes a valuable resource not only for teaching but also for learning and practicing Mathematics. Thus, future research can expand knowledge about the problem-solving methodology, exploring its application in different contexts and content, developing new teaching resources, focusing on teacher training, and comparing it with other innovative approaches.

		\vspace{\onelineskip}

		\noindent
		\textbf{Key-words}: Problem Solving Methodology; Spatial Geometry; Polyhedrons.
	\end{otherlanguage*}
\end{resumo}
