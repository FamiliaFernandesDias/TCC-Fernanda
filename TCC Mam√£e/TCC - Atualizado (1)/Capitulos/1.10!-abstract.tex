
% resumo em inglês
\begin{resumo}[Abstract]
	\begin{otherlanguage*}{english}
		It is observed that, in the teaching of Spatial Geometry, many students have difficulties in understanding abstract concepts, such as those related to polyhedra, and in applying them in practical contexts. The traditional approach, often focused on memorizing formulas and procedures, does not seem to promote a deep and meaningful understanding of the content. Therefore, the objective of this work is to verify the effectiveness of the Teaching-Learning-Assessment Methodology of Mathematics through Problem Solving in the construction of Spatial Geometry concepts for 2nd-year high school students in public schools. The data collection procedures were the observation and recording of students' productions generated in the development of each problem situation addressing the concepts of polyhedra and non-polyhedra, face, edge, vertex (Euler's Relation), lateral area, total area, and volume of polyhedra, lateral area, total area, and volume of pyramids. It is an educational approach in which the pioneer in valuing problem-solving was \citeonline{polya1978}, and his proposal was to make Mathematics students good problem solvers. There have been advances and setbacks regarding this methodology, and many have contributed to this, including \citeonline{van_de_walle_elementary_2000} and \citeonline{resolucaoDeProblemas2014}, placing the student at the center of the learning process, encouraging them to be active and autonomous learners. The theoretical assumption that underpins the research was supported by the stages of the Problem Solving methodology in the Teaching-Learning-Assessment process proposed by \citeonline{BOLEMAAllevatoOnuchic2011}. The analysis results indicate that the Problem Solving Methodology in Teaching-Learning-Assessment can constitute a differential for the teaching of Polyhedra in Spatial Geometry, as it fosters a motivating and challenging environment while providing the necessary conditions for the active involvement of the student in the process of constructing their knowledge. It can also be seen that the Problem-Solving Methodology can contribute both to the development of geometric thinking and reasoning, aiming at a significant evolution of qualitative and quantitative performance.

		\vspace{\onelineskip}

		\noindent
		\textbf{Key-words}: Problem Solving Methodology; Spatial Geometry; Polyhedrons.
	\end{otherlanguage*}
\end{resumo}
