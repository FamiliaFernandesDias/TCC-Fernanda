
% resumo em português
\setlength{\absparsep}{18pt} % ajusta o espaçamento dos parágrafos do resumo
\begin{resumo}

	Percebe-se que, no ensino da Geometria Espacial, muitos alunos apresentam dificuldades em compreender conceitos abstratos, como os relacionados aos poliedros, e em aplicá-los em contextos práticos. A abordagem tradicional, muitas vezes focada na memorização de fórmulas e procedimentos, não parece promover uma compreensão profunda e significativa dos conteúdos. Assim sendo, o objetivo deste trabalho é verificar a eficácia da Metodologia de Ensino-Aprendizagem-Avaliação de Matemática através da Resolução de Problemas na construção de conceitos de Geometria Espacial para alunos da 2ª série do Ensino Médio da escola pública. Os procedimentos de coleta de dados foram a observação e o registro das produções dos alunos geradas no desenvolvimento de cada situação-problema abordando os conceitos de poliedros e não poliedros, face, aresta, vértice (Relação de Euler), área lateral, total e volume dos prismas, área lateral, total e volume das pirâmides. Trata-se de uma abordagem educacional em que o pioneiro a valorizar a resolução de problemas foi \citeonline{polya1978} e, sua proposta era tornar os estudantes de Matemática bons resolvedores de problemas. Houve avanços e recuos em relação a essa metodologia e muitos colaboraram para isto, entre eles: \citeonline{van_de_walle_elementary_2000} e \citeonline{resolucaoDeProblemas2014}, colocando o aluno no centro do processo de aprendizagem, incentivando-o a ser um aprendiz ativo e autônomo. O pressuposto teórico que fundamenta a pesquisa foi apoiado nas etapas da metodologia de Resolução de Problemas no processo de Ensino-Aprendizagem-Avaliação proposta por \citeonline{BOLEMAAllevatoOnuchic2011}. Os resultados da análise dão indicativo de que a Metodologia Resolução de Problemas no Ensino-Aprendizagem-Avaliação pode constituir um diferencial para o ensino de Poliedros da Geometria Espacial, pois favorece um ambiente motivador e desafiador, ao mesmo tempo em que propicia condições necessárias para o envolvimento ativo do aluno no processo de construção de seu conhecimento. Também pode-se perceber que a Metodologia da Resolução de Problemas pode contribuir tanto para o desenvolvimento do pensamento e raciocínio geométricos, com vistas a uma evolução significativa dos rendimentos qualitativos e quantitativos.

	\textbf{Palavras-chave}: Metodologia Resolução de Problemas; Geometria Espacial; Poliedros.

\end{resumo}
