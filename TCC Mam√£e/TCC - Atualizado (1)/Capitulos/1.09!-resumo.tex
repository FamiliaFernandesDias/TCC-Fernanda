
% resumo em português
\setlength{\absparsep}{18pt} % ajusta o espaçamento dos parágrafos do resumo
\begin{resumo}
  % A metodologia ativa de resolução de problemas é uma abordagem educacional que coloca os alunos no centro do processo de aprendizagem, incentivando-os a resolver problemas do mundo real por meio da aplicação de conhecimentos e habilidades. Esta metodologia é particularmente eficaz pois promove a participação ativa, o pensamento crítico, a criatividade e a colaboração. A pesquisa busca responder se essa metodologia constitui um caminho alternativo eficaz para a construção de conceitos geométricos espaciais. Os objetivos incluem desenvolver habilidades de resolução de problemas, compreensão profunda de conceitos geométricos, aplicação prática da geometria, desenvolvimento de competências transversais e aumento do engajamento e motivação dos alunos. Os professores atuam como facilitadores, guiando os alunos através de atividades práticas e colaborativas, como estudos de caso, simulações, projetos de grupo, debates e experiências práticas. A metodologia inclui etapas como identificação do problema, contextualização, exploração, colaboração, experimentação, reflexão e apresentação. A dissertação está organizada em capítulos que abordam a introdução, referencial teórico, aspectos metodológicos, desenvolvimento das atividades didáticas e considerações finais.

  % A metodologia ativa de resolução de problemas é uma abordagem educacional que coloca os alunos no centro do processo de aprendizagem, incentivando-os a resolver problemas do mundo real por meio da aplicação de conhecimentos e habilidades. Neste trabalho, propõe-se a aplicação dessa metodologia para a construção do aprendizado sobre conceitos geométricos espaciais com alunos do ensino médio. Para tanto, o professor atua como facilitador, elaborando atividades que promovam a participação ativa, o pensamento crítico, a criatividade e a colaboração dos alunos. Nessas aulas especialmente desenvolvidas, segue-se uma sequência de dez etapas que incluem a leitura do problema, coletiva, solução do problema pelos docentes acompanhados pelo docente e posteriormente apresentados aos demais alunos com os quais as resoluções eram discutidas para que a turma alcançasse um consenso, por fim formalizando o conteúdo abordado. Com a aplicação dessa metodologia, foi identificado elevada taxa de participação dos alunos, e um aumento significativo no engajamento e motivação dos discentes. Além disso, através do feedback dado pelos alunos, demonstra-se a satisfação dos mesmos com a metodologia aplicada. Diante do exposto, conclui-se que a metodologia ativa de resolução de problemas é uma abordagem eficaz para a construção de conceitos geométricos espaciais, visto que promove o engajamento e motivação nos alunos, bem como os auxilia a fixar o conteúdo de forma mais eficaz.

  Percebe-se que, no ensino da Geometria Espacial, muitos alunos apresentam dificuldades em compreender conceitos abstratos, como os relacionados aos poliedros, e em aplicá-los em contextos práticos. A abordagem tradicional, muitas vezes focada na memorização de fórmulas e procedimentos, não parece promover uma compreensão profunda e significativa dos conteúdos. Assim sendo, o objetivo do presente trabalho é verificar se o uso da Metodologia de Ensino-Aprendizagem-Avaliação de Matemática através da Resolução de Problemas \cite{resolucaoDeProblemas2019, polya1978} constitui-se num caminho alternativo para a construção de conceitos e conteúdos geométricos espaciais pelos alunos do Ensino Médio. Trata-se de uma abordagem educacional que coloca o aluno no centro do processo de aprendizagem, incentivando-o a ser um aprendiz ativo e autônomo \cite{polya1978}. Nesse método, os alunos aprendem através da resolução de problemas reais ou simulados, em vez de simplesmente receber informações passivamente. Esta abordagem foi aplicada a estudantes da 2º série do Ensino Médio da escola pública. Ao final desta pesquisa conseguimos demonstrar que a resolução de problemas é um dos principais métodos para ensinar, aprender e avaliar a Matemática em sala de aula. Com base nas evidências coletadas nesta pesquisa, acreditamos fortemente que a Metodologia de Ensino-Aprendizagem-Avaliação de Matemática através da Resolução de Problemas é uma alternativa eficaz que permite aos alunos a construção de conceitos e conteúdos matemáticos, explorando e aproveitando seu próprio potencial e habilidades. Os alunos ficaram comprometidos com o trabalho e focados nas atividades propostas, não havendo dispersão. Conseguimos também, uma evolução significativa dos rendimentos qualitativos e quantitativos. Diante dos resultados é possível concluir que a Metodologia Resolução de Problemas se torna, assim, um recurso valioso não só para ensinar, mas também para aprender e praticar Matemática. Assim, futuras pesquisas podem expandir o conhecimento sobre a metodologia de resolução de problemas, explorando sua aplicação em diferentes contextos e conteúdos, desenvolvendo novos recursos didáticos, focando na formação de professores e comparando-a com outras abordagens inovadoras.

  \textbf{Palavras-chave}:  Metodologia Resolução de Problemas; Geometria Espacial; Poliedros.

\end{resumo}
